\chapter{Grundlagen}
\label{grundlagen}
\section{Erklärende Begriffe}
\label{sec:begriffe}
Ein Programm ist die Beschreibung eines mechanischen Rechenverfahrens, dass der Computer speichern und ausführen kann \parencite[vgl.][14]{rechenberg2000}. Vergleichbar ist dies mit einem Rezept. Wenn das Programm ausgeführt wird, nennt man es dann einen Prozess \parencite[vgl.][S.71]{tanenbaum2016}. Das Programm löst sozusagen einen Prozess aus. In der Analogie entspricht dies dem Koch, welcher das Rezept ausführt. Dabei braucht ein Prozess auch “Zutaten”, um das Programm auszuführen, diese werden Betriebsmittel genannt. Das entspricht in der Realität zum Beispiel Arbeitsspeicher oder Dateien. Somit sind Betriebsmittel Ressourcen, die beschrieben werden können. Es gibt wiederverwendbare Betriebsmittel oder konsumierbare Betriebsmittel. Die wiederverwendbaren Betriebsmittel werden von Prozessen belegt und nach deren Ausführung zur nächsten Verwendung  wieder freigegeben. Konsumierbare Betriebsmittel werden im laufenden System erzeugt (produziert) und anschließend zerstört (konsumiert).


Die Benutzeroberfläche des Betriebssystems ist die Schnittstelle zwischen Hard- und Software. An dieser lässt sich beobachten, dass die Prozesse auf den Kernen des Prozessors sehr schnell wechseln. In wenigen Nanosekunden wechselt die CPU also zwischen den Prozessen umher, sodass zu einem Zeitpunkt nur ein Prozess läuft. Durch das schnelle Umschalten sieht es aus als würden mehrere Prozesse gleichzeitig laufen \parencite[vgl.][S.466]{sommer2002}. Damit  das funktioniert, wird ein Scheduler benötigt, welcher die Prozesse in eine Reihenfolge bringt und entscheidet welcher Prozess wann rechnen darf \parencite[vgl.][S.199]{tanenbaum2016}.

Es kann allerdings auch passieren, dass ein Scheduler selbst einen verklemmten Prozess weiter rechnen lässt, was allerdings zu keinem Ergebnis führt.
Wenn nun zwei Prozesse bereit zum Rechnen sind und um das gleiche Betriebsmittel konkurrieren und das Endergebnis der zwei Prozesse davon abhängt, welcher von den Prozessen als erstes rechnen darf, wird schnell der kritische Abschnitt erreicht. Der kritische Abschnitt umfasst \glqq die Teile des Programms, in denen auf gemeinsam genutzter Speicher zugegriffen wird'' \parencite[S. 164]{tanenbaum2016}. Wenn dieser kritische Abschnitt betreten wird, kommt es zu sogenannten Race Conditions. Race Conditions sind Situationen, \glqq in denen zwei oder mehr Prozesse einen gemeinsamen Speicher lesen oder beschreiben und das Endergebnis davon abhängt, welcher wann genau läuft'' \parencite[S. 166 f.]{tanenbaum2016}. Diese Situationen müssen vermieden werden, wenn von zwei Prozessen die gleiche Datei beschrieben werden soll und es davon abhängt wann welcher Prozess rechnen darf, zum Beispiel wenn ein Dokument bedruckt werden soll. Beschreiben zwei Prozesse das gleiche Dokument so kann es passieren kein inhaltlicher Zusammenhang mehr besteht. Verhindert werden kann dies mit wechselseitigem Ausschluss. Es kommt dann also gar nicht dazu, dass zwei Prozesse das gleiche Betriebsmittel verwenden bzw. gleichzeitig den kritischen Abschnitt betreten \parencite [vgl][S.471]{sommer2002}.

\section{Begriff Verklemmungen}
\glqq Eine Verklemmung (Deadlock)bezeichnet einen Zustand, in dem die beteiligten Prozesse wechselseitig auf den Eintritt von Bedingungen warten, die nur durch andere Prozesse aus dieser Gruppe selbst hergestellt werden können.'' \parencite[S.248]{sturm2001}. Diese wird durch Synchronisationsfehler erzeugt. 
Demnach macht keiner der Prozesse einen Fortschritt.

Es benötigt insgesamt vier Voraussetzungen damit letztendlich eine Verklemmung entsteht. Die erste ist der wechselseitige Ausschluss der Prozesse miteinander (engl. mutual exclusion). Dadurch ist ein Betriebsmittel unteilbar und exklusiv nutzbar. So werden zwar Race Conditions vermieden aber es besteht nun die Voraussetzung, dass zwei Prozesse nicht zur selben Zeit das gleiche Betriebsmittel verwenden können. Die Zweite ist das Nachfordern von den Betriebsmitteln ohne ein anderes loszulassen (engl. hold and wait). Die konkurrierenden Prozesse können nur schrittweise die Betriebsmittel belegen. Die dritte Bedingung besteht aus dem Fakt, dass einem Prozess die Betriebsmittel nicht entzogen werden können und nicht rückforderbar sind (engl. no preemption). Damit überhaupt eine Verklemmung vorliegen kann, müssen alle diese drei Bedingungen und eine Weitere eintreten. Diese vierte Bedingung heißt auf Englisch \grqq circular wait\grqq und wird mit \glqq zirkuläres Warten\grqq übersetzt. Deshalb muss es eine geschlossene Kette an wartenden Prozessen geben. Um zu einem Ergebnis zu kommen benötigt Prozess A das, was Prozess B erst herstellen muss und Prozess B, das was Prozess A erst noch herstellen muss. Also kommt es zu einem wechselseitigen kreisförmigen(zirkulären) Warten \parencite[vgl.][S. 195f.]{baun2017}.

Neben dem Deadlock, bei dem der Prozesszustand  \inline{BLOCKED} ist, gibt es auch noch den Livelock. Bei diesem ist der Prozesszustand \inline{RUNNING}, dabei führt die CPU den Prozess gar nicht aus da dieser mit einem anderen verklemmt ist. Der Livelock ist also wesentlich schwerer zu erkennen als der Deadlock \parencite[vgl.][S. 561 f.]{tanenbaum2016}.