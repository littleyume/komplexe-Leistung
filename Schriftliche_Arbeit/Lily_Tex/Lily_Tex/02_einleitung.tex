\chapter{Einleitung}
\label{sec:Einleitung}
Stellen Sie sich vor, Sie fahren auf einer Straße in ihrem Auto und gelangen an eine Kreuzung ohne Ampeln und Verkehrsschilder. Sie stehen mit vier weiteren Autos an dieser Kreuzung. In diesen Situationen gilt die Regel „rechts vor links“. Das bedeutet, dass kein Auto fahren darf, solange nicht das Auto rechts von ihm von der Stelle gefahren ist. Bestenfalls fahren zwei Autos gleichzeitig los und stoppen wieder, um einen Aufprall zu verhindern, das ist eine Frage der Koordination.
Abstrahiert man dieses Problem in die IT und ersetzt die Autos durch Prozesse, wird es Verklemmung genannt.
Die folgenden Seiten werden konkret das Modell der speisenden Philosophen genauer beleuchten.
Dieses Problem wurde als erstes 1965 von Djikstra formuliert und von demselben auch gelöst. Seidem ist es ein beliebtes Beispiel zur Visualisierung bzw. Illustration von Deadlocks. Es ist eines der Worst-Case-Szenarios um Probleme, in diesem Fall Deadlocks, in Algorithmen zu zeigen.

Ein Beispiel eines Deadlocks gab es im Jahr 2019. Microsoft hatte ein Sammelupdate für Windows 10 herausgebracht, welches ein paar Probleme lösen sollte. Allerdings hatte sich bei Benutzern, die Windows 10 Version 1903 benutzten und das Update installierten, ein Prozess von Cortana verklemmt. Das Problem trat dann auf, wenn eine Anfrage an die Suchmaschine Bing von einer Methode wie einer Registry-Anfrage unterbunden wurde. Das Ergebnis dieses Deadlocks war, dass die Suchmaschine keine Suchergebnisse angezeigt hat sowie dass die CPU-Auslastung dauerhaft bei 40 Prozent lag \parencite[vgl.][]{bug}.
 
Zu Beginn dieser Arbeit werden Grundlagen von Prozessen, Programmen sowie des Schedulings nachfolgend vorgestellt. Das darauf folgende Kapitel wird eine Verklemmung definieren, sowie die Bedingungen für eine solche erklären. Lösungsansätze werden vorgestellt, sowie Möglichkeiten gezeigt, wie eine Verklemmung visualisiert werden kann. Lösungsansätze einer Verklemmungssituation sind die Themen des letzten Kapitels. Somit wird mit der Programmierung des Modells und dessen Lösung diese Arbeit beendet.  
 