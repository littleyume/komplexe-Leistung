\chapter{Einleitung}
\label{sec:Einleitung}
Stellen Sie sich vor, Sie fahren auf einer Straße in ihrem Auto und gelangen an eine Kreuzung ohne Ampeln und Verkehrsschilder. Sie stehen mit vier weiteren Autos an dieser Kreuzung. In diesen Situationen gilt die Regel „rechts vor links“. Das bedeutet, dass kein Auto fahren darf, solange nicht das Auto rechts von ihm von der Stelle gefahren ist. Bestenfalls fahren zwei Autos gleichzeitig los und stoppen wieder, um einen Aufprall zu verhindern, das ist eine Frage der Koordination.
Abstrahiert man dieses Problem in die IT und ersetzt die Autos durch Prozesse, nennt man dies eine Verklemmung.
Die folgenden Seiten werden konkret das Problem der speisenden Philosophen genauer beleuchten.
Dieses Problem wurde als erstes 1965 von Djikstra formuliert und von dem selben auch gelöst. Seitem ist es ein beliebtes Beispiel zur Visualisierung bzw. Illustration von Verklemmungen und Deadlocks. Es ist eines der Worst-Case-Szenarios um Probleme in Algorithmen zu zeigen, in diesem Fall ist das Problem somit ein Deadlock.

Ein Beispiel eines Deadlocks gab es im Jahr 2019. Microsoft hatte ein Sammelupdate für Windows 10 herausgebracht, welches ein paar Probleme lösen sollte. Allerdings hatte sich bei Benutzern, die Windows 10 Version 1903 benutzten und das Update installierten, ein Prozess von Cortana verklemmt. Dabei kam es erst zu einer dauerhaften CPU-Auslastung von  ca 40 Prozent, da das Problem dann auftrat, wenn eine Anfrage an die Suchmaschine Bing von einer Methode wie einer Registry-Anfrage unterbunden wurde. Das Ergebnis dieses Deadlocks war, dass die Suchmaschine keine Suchergebnisse angezeigt hat.\parencite[vgl.][]{bug}

Es gibt unzählige Lösungen des Philosophen-Problems bis heute. 
Zu Anfang werden Grundlagen von Prozessen, Programmen sowie des Schedulings vorgestellt. Das darauf folgende Kapitel wird eine Verklemmung definieren, sowie die Bedingungen für eine solche erklären. 
Lösungsansätze werden vorgestellt, sowie Möglichkeiten gezeigt, wie eine Verklemmung visualisiert werden kann. Lösungsansätze einer Verklemungssituation sind die Themen des letzten Kapitels.