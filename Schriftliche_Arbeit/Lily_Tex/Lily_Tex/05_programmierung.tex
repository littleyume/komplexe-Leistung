\chapter{Programmierung}
\label{programmierung}
\section{Erste Lösung mit Problemdarstellung}
\label{erste_lösung}

In diesem Kapitel soll die mögliche Programmierung des Modells der speisenden Philosophen mithilfe der Programmiersprache Python3 vorgestellt werden. Das Programm wurde in Sinnabschnitte geteilt und abschnittsweise beschrieben.

In dem Programm werden die Bibliotheken \glqq Multiprocessing'' sowie \glqq random'' und \glqq time'' importiert. Die Klasse \glqq Philosophen'' erbt von der Klasse \glqq Process'' alle Funktionen und erhält zusätzlich die, die in der Klasse \glqq Philosophen'' definiert werden. Mit der Klasse \glqq Process'' kann ein Subprozess initiiert werden. Da sich die Philosophen über die gleichen Merkmale definieren, können sie in einer Klasse zusammengefasst werden. Dem Konstruktor werden die Parameter \inline{name} sowie \inline{leftFork} und  \inline{rightFork} übergeben. Aus letzteren werden die Objekte leftFork und rightFork erzeugt. Diese stehen für die rechte sowie linke Gabel. Mit dem Parameter \inline{name} kann dem Prozess ein Name gegeben werden. \glqq Diese Methode wird nur zu Identifikationszwecken genutzt\grqq \parencite{lock}.

\begin{lstlisting}[style = Python, label = {Code_erste_lösung}, caption = {Klasse Philosophen}]

from multiprocessing import Process, current_process, RLock
import time
import random


class Philosophen(Process):
    def __init__(self, name, leftFork, rightFork):
        print("{} hat sich an den Tisch gesetzt".format(name))
        Process.__init__(self, name=name)
        self.leftFork = leftFork
        self.rightFork = rightFork
        
\end{lstlisting}

Die von der Klasse \glqq Process \grqq geerbte \inline{run()} -Funktion wurde in das Programm eingefügt um die Handlungsfolge der Tätigkeiten zu bestimmen \parencite[vgl.]{lock} . In der Zeit, in der der Philosoph denkt, schläft der Prozess (Programmzeile 7) repräsentativ. Für den angegebenen Zeitraum wird die Ausführung des Programms angehalten. Ist diese Zeitspanne vorbei, so geht der Programmcode weiter. In diesem Beispiel wird der besprochene Zeitraum zufällig mithilfe der \inline{random} Methode zwischen einer und fünf Sekunden ausgewählt \parencite[vgl.]{sleep}. Wenn der Philosoph fertig mit Denken ist, fordert er die linke Gabel an. Danach soll er versuchen, die rechte Gabel zu bekommen. Hat er beide Gabeln so kann er essen. Ist er damit fertig, so kann er erst die rechte und dann die linke Gabel wieder freigeben.

\begin{lstlisting}[style = Python, caption = {erste Lösung}]

    def run(self):
        print("{} hat angefangen zu denken"
        .format(current_process().name)) 
        #Philosoph x hat mit denken begonnen
        while True:
            time.sleep(random.randint(1, 5))
            print("{} ist fertig mit denken"
            .format(current_process().name))
            self.leftFork.acquire() 
            #philosoph x hat die linke Gabel
            time.sleep(random.randint(1, 5))
            try:
                print("{} hat die linke Gabel bekommen"
                .format(current_process().name))

                self.rightFork.acquire() 
                #Philosoph x hat die rechte Gabel
                try:
                    print("{} hat beide Gabeln, isst gerade"
                    .format(current_process().name)) 
                    #Philosoph x hat beide Gabeln

                finally:
                    self.rightFork.release()
                    print("{} hat die rechte Gabel freigegeben"
                    .format(current_process().name)) 
                    #Philosoph ist fertg mit essen und 
                    #gibt die rechte Gabel frei

            finally:
                self.leftFork.release()
                print("{} hat die linke Gabel freigegeben"
                .format(current_process().name)) 
                #Philosoph x hat gibt die linke Gabel frei

\end{lstlisting}

Nun der Lösungsansatz hierbei ist, dass die Gabeln durch einen RLock repräsentiert werden. Das heißt die Prozesse besetzen immer Sperren. Um die Sperren zu besetzen, ruft der Prozess die \inline{aquire()}-Funktion auf. Um sie wieder freizugeben ruft er die \inline{release()}-Funktion auf. Diese \glqq Lösung\grqq ist jedoch nicht deadlockfrei. Wollen zwei Prozesse die gleiche Sperre besetzen so verklemmt das Programm sofort. 

\section{Lösungen}
\label{endloesung}

Um diesem Problem vorzubeugen kann man der \inline{aquire()}-Funktion über den Parameter \inline{blocking} ein True oder ein False mitgeben. Setzt man blocking auf True, so geht der Programmcode nicht weiter bis die Funktion ausgeführt ist. In diesem Fall wenn der Lock besetzt ist. Das heißt, will ein Prozess den Lock besetzen, so muss er immer wieder versuchen, diesen zu besetzen bis dieser wieder freigegeben ist \parencite[vgl. ]{lock}. Der Prozess wird also immer wieder versuchen die linke Gabel zu belegen, bis diese freigegeben ist. 

\begin{lstlisting}[style = Python, caption = {Endlösung}]

    def run(self):
        print("{} hat mit denken begonnen"
        .format(current_process().name)) 
        #Philosoph x hat mit denken begonnen
        while True:
            time.sleep(random.randint(1, 5))
            print("{} ist fertig mit denken"
            .format(current_process().name)) 
            #Philosoph x ist bereit zum Essen
            self.leftFork.acquire(True) 
            #prozess x will den Lock belegen und muss es so 
            #lange versuchen bis er frei ist
            time.sleep(random.randint(1, 5))

\end{lstlisting}

Dann soll er versuchen die rechte Gabel zu erlangen. Da dieser RLock auf False gesetzt ist, wird er das nur einmal versuchen \parencite[vgl.]{lock}.

Zusammengefasst gibt es blockierende Funktionen, bei denen der Prozess solange den Lock zu besetzen versucht bis dieser frei ist und es gibt nichtblockierende Funktionen, welche nur einmal versuchen den Lock zu besetzen \parencite[vgl.][S.164]{sturm2001}. Geprüft werden kann dies an der Variable \glqq locked''. Scheitert dieser Versuch, so ist der Prozess nicht auf \glqq locked'' gesetzt und die linke Gabel wird wieder freigeben. Ist der Zustand des RLock's auf locked, so ist der Versuch gelungen.  Der Philosoph hat nun beide Gabeln, kann essen, dann die rechte und zuletzt auch die linke Gabel wieder freigeben. Nun kann der nächste Prozess wieder versuchen den Lock zu besetzen.

\begin{lstlisting}[style = Python, label = {rechte_gabel}, caption = {nichtblockierender Code bei Versuch die rechte Gaabel anzufordern}]

try:
                print("{} hat die linke Gabel"
                .format(current_process().name))
                locked = self.rightFork.acquire(False) 
                #Prozess x versucht nur einmal den lock zu besetzen
                if locked:
                    print("{} hat beide Gabeln, isst gerade"
                    .format(current_process().name)) 
                     #Prozess x hat beide Locks besetzt
                    self.rightFork.release() 
                    print("{} hat die rechte Gabel freigegeben"
                    .format(current_process().name)) 
                    self.leftFork.release() 
                    print("{} hat die linke Gabel freigegeben"
                    .format(current_process().name))
                    #wenn Zustand des Prozesses locked ist
                    #(die Philosophen gegessen haben),
                    #wird erst der Lock "rightFork" 
                    #und dann Lock "leftFork" freigegeben
                else:
                    self.leftFork.release() 
                    #wenn Zustand des Prozesses nicht locked ist,
                    #dann wird Lock "leftFork"freigegeben
                    #("rightFork" war  nicht frei)
                    print("{} hat die linke Gabel freigegeben"
                    .format(current_process().name))  


\end{lstlisting}
 







