\chapter{Fazit}
\label{fazit}

Zusammenfassend muss gesagt werden, dass eine Verklemmung (= Deadlock) nur unter vier Bedingungen entstehen kann. Sie kann unter dem Modell der speisenden Philosophen veranschaulicht werden. Es gibt verschiedene Arten eine Verklemmung zu visualisieren, wie zum Beispiel der Betriebsmittelbelegungsgraph. Dabei gibt es auch verschiedene Möglichkeiten auf eine Verklemmung zu reagieren. 

?????????????Zum einen das Ignorieren des Problems, wobei durch das verklemmen der Prozesse diese nicht terminieren, das Erkennen und Beseitigen der Verklemmungen, die Vermeidung sowie den Ausschluss einer Verklemmung, indem der Eintritt einer der vier Bedingungen verhindert wird.????????????????????

Es gab naive Lösungsansätze, die zumindest nicht verklemmen, allerdings..... Verlieren Effekt von paralleler Programmierung.... Vermittler und Mutex für gabeln und essen.

Arbeit unter betracht einprozessorsystem wie in \ref{sec:begriffe} \nameref{sec:begriffe}  dargestellt wurde1, mittlerweile mehrprozessorsystem, selbes prinzip